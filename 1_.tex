\documentclass{article}
\usepackage[utf8x]{inputenc}
\usepackage[T2A]{fontenc}
\usepackage[russian]{babel}
\usepackage{indentfirst}
\usepackage{longtable}
\usepackage{amsmath,accents}
\usepackage[left=2cm, right=2cm, top=2cm, bottom=2cm]{geometry}

\begin{document}
\textbf{\huge{Условный вывод при ограничении одновременного стохастического упорядочения}}

\section*{Аннотоция}
Тестирование стохастического упорядочения имеет большое значение при сравнениии увеличения доз лечения, но в приложениях, включающих многофакторные ответы, получило намного меньше внимания. Мы предлагаем перестановочный тест для проверки многофакторного стохастического упорядочения. Этот тест не зависит от распределения и не требует предположений о зависимости между переменными. Сравнительное имитационное исследование показывает, что предложенное решение показывает хорошую общую производительность при сравнении с существующими тестами, которые могут быть применены для той же задачи.

\section{Введение}
В эксперименте типа доза-ответ $c$ доз лечения распределяются по независимым группам субъектов. Пусть $X$ это вектор ответа $p$ факторов для $i$-го субъекта, случайно распределенного в дозу $j$, и пусть общее число наблюдений $n$.

Предположим, что $X$ это $n$ независимых одинаково распределенных случайный векторов с непрерывной функцией распределения, определенной на $R^p$ и существует конечное математическое ожидание, также обозначим через $F$ $h$-ю маргинальное распределение для $F$.

Распространенная структура, часто используемая при сравнении увеличения доз лечения, предполагает, что $c$ распределений являются стохастически упорядоченными.

Вывод, основанный на стохастически упорядоченных однофакторых случайных величинах был широко исследован, в то время какстохастически оупорядоченные случайные вектора получили намного меньше внимания.

$c$ многофакторных распределений называются стохастически упорядоченными тогда и только тогда, когда "неравенство" выполняется для всех возрастающих функций таких, что математическое ожидание существует.

Мы хотим проверить нулевую гипотезу, где "равенство" означает равенство распределений, против альтернативной гипотезы.

Отметим, что когда применяется проверка гипотез, предполагается, что либо $H_0$, либо $H_1$ выполняется, таком образом мы должны делать априорное предположение, что (1) выполняется. Из следствия 3 следует, что при (1) $H_0$ выполняется в том и только том случае, когда $X$ имеют одинаковые маргинальные распределения, а именно ... .

Как следствие, $H_1$ выполняется тогда и только тогда, когда "неравенство выполняется" и $X$ не равны в распределениях, выражается как ..., при хотя бы 1 строгом неравенстве, выполняющемся для какого-то $x$ хотя бы для какого-то $h$.

В составнsх задачах, как обсуждалось, предпочтительно смотреть на $H_0$ как на пересечение более простых гипотез и на $H_1$ как на объединение того же числа соответствующих простых альтернатив.

В непараметрической постановке, пусть $F$ будет классом всех непрерывных функций распределния на $R^p$, тогда модель одновременного стохастического упорядочения случайных величин "записывается так".

Если мы предполагаем, что распределения $F$ могут отличаться только в отношении к положению, т.е. ..., тогда предыдущая модель упрощается до гомоскедастичной модели положения ..., где $a<b$ значит ... .

Пусть $F$ это класс всех $p$-факторных нормальных распределений; нормальная гомоскедастичная модель положения соответствует ... . Очевидно, ... . При гомоскедастичной модели положения задача тестирования сводится к ... против ... .

\subsection{Дальнейшая декомпозиция задачи}
Рассмотрим $h$-ю подпроблему проверки $H_0$ против $H_1$. Непараметрическое ранговое рашение этой задачи дается тестом Джонски-Тепстры. С другой стороны, 2005 предложил ассимптотически тест с использованем процедуры последовательного тиестирования, разработанную 1965. Эта процедура состоит в последовательном тестировании гипотез как двухвыборочного теста объединяя первые $(k-1)$ выборок для тестирования $H_0$, где "неравенство" значит "неравенство" со строгим неравенством, выполняющимся для какого-то $x$. Таким образом, гипотезы могут быть выражены через ... соответственно. Однако, гипотезы могут быть также выражены, как ... , где для тестирования гипотезы мы переходим к задаче двухвыборочного тестирования объединяя первые $(k-1)$ и последние $(c-k+1)$ выборок.

Аналогично, при  гомоскедастичной модели расположения, мы можем разложить гипотезы на ... . Таким образом, мы можем выразить нулевую гипотезу как ... и альтернативу как ... .

\section{Объединение зависимых перестановочных тестов}
Теперь рассмотрим общую задачу одновременного тестирования конечного числа гипотез. Будем предполагать, что тесты для отдельной гипотезы доступны и задача заключается в том, как объединить их в одновременную тестовую процедуру.

Так как в общем рассматриваемые тесты и соответствующие $p$-значения положительно зависимы, не допускается использование методов объединения независимых $p$-значений.

Простой способ - проводить общий тест с процедурой Бонферрони, которая потенциально консервативна, т.к. она основана на неравенстве Бонферрони и т.к. она игнорирует потенциальную завиимость между отдельными выводами.

Следовательно, мы будем использовать метод непараметрического объединения в рамках перестановочной парадигмы для объединения частичных тестов в общий тест. За деталями метода непараметрического комбинирования мы ссылаемся на 2001.

В перестановочном контексте обозначим набор данных в представлении 1 к 1-му, где подразумевается, что первые $n_1$ векторов в списке принадлежит первой выборке, следующие $n_2$ - ко второй и т.д. Определим перестановочное выборочное пространство как мнодество, содержащее все $n!$ перестановок наблюдаемого набора данных. Рассмотрим $b$-ю перестановку, тогда ... обозначает $b$-ю перестановку и ... обозначает $c$-ю перестановочную выборку, где ... .

Для того, чтобы увидеть как объединить зависимые перестановочные тесты для проверки $H_0$ против $H_1$, предположим, что частичный перестановочный тест основывается на тестовой статистике $T$, для которой большие значения значимы. Обозначим величину тестовой статистики, посчитанной на $X$ как ... и ... соответственно.

(условное) частичное $p$-значение относящееся к тесту ..., где ... обозначает функцию-индикатор. Пусть ... перестановочное нулевое распределение выживания.

Непараметрическая комбинация реализуется в 2 стадии:
\begin{itemize}
  \item На первом шаге для каждой компоненты переменной, комбинация составляется относительно $(c-1)$ частичных тестов, тогда мы получаем объединенный перестановочный тест первого порядка ... , где ... является допустимой комбинативной функцией. $p$-значение первого порядка, относящееся к перестановочному тесту ... будетперестоновоным нулевым распределением выживания.

  \item На втором шаге, комбинирование выполняется относительно $p$ переменных, тогда мы получаем общий тест (второго порядка) ..., где ... - допустимая комбинативная функция, которая не обязательно совпадает с предыдущей.
\end{itemize}

Следовательно, нулевая ипотеза отвергается при уровне значимости если ..., т.к. при нулевой гипотезе все перестановки равновероятны, тогда ... удовлетворяет ... для всех ..., где ... .

Когда мощность велика, мы можем рассматривать его как случайную выборку (с или без возвращений), тогда .. может быть оценено с желамой степенью точности.

Еще одной особенностью этого метода является то, что когда общий анализ отвергает $H_0$, можно применить множественную процедуру тестирования чтобы найти какой частичный тест (или группа тестов) наиболее ответственна в глобальном отклонении. Тогда, после поправки $p$-значения на множественность (например, используя метод закрытого тестирования), мы можем выделить какое подмножество переменных представляет статистически значимое упорядочение,таким образом, какие ... .

\subsection{Выбор комбинирующей функции}


\end{document}
